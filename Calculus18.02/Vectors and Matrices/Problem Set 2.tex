%%
%% Author: maciejsrokowski
%% 27/07/2018
%%

% Preamble
\documentclass[11pt]{article}

% Packages
\usepackage{amsmath}
\usepackage{physics}
\usepackage{siunitx}
\usepackage{graphicx}
\usepackage{cancel}
\usepackage{a4wide}

% Document
\begin{document}

    \title{Problem Set 2}
    \maketitle

    \newtheorem{theorem}{Theorem}
    \newtheorem{definition}{Definition}
    \newtheorem{example}{Example}
    \newtheorem{problem}{Problem}

    \section{Matrices and inverse matrices}
    \begin{problem}
        \textbf{1F 5b}
        Find $A^2 ,A^3 , A^n $ if $A=\bmqty{1 & 1 \\ 0 & 1}$
        \begin{gather*}
            A^2 = \bmqty{1 & 2 \\ 0 & 1}\\
            A^3 = \bmqty{1 & 3 \\ 0 & 1}\\
            \text{Guess:}\\
            A^n = \bmqty{1 & n \\ 0 & 1}\\
        \end{gather*}
    \end{problem}
    \begin{problem}
        \textbf{1F 8a}
        If \[
               A \bmqty{1 \\ 0 \\ 0} = \bmqty{2 \\ 3 \\ 1 },
               A \bmqty{0 \\1 \\ 0 } = \bmqty{-1 \\ 0 \\ 4 },
               A \bmqty{0 \\ 0 \\ 1} = \bmqty{1 \\ 1 \\ -1}
        \], what is the matrix $3 \cross 3 $ A\\
        \begin{gather*}
            A = \bmqty{
            2 & -1 & 1 \\
            3 & 0 & 1 \\
            1 & 4 & -1
            }
        \end{gather*}
    \end{problem}
    \begin{problem}
        \textbf{1G 3}\\
        \begin{gather*}
            A = \bmqty{1 & -1 & 1\\
            0 & 1 & 1 \\
            -1 & -1 & 2}\\
            b = \bmqty{ 2 \\ 0 \\ 3 }\\
            \text{Solve $Ax=b$ by finding $A^{-1}$}\\
            cofactors(A)=\bmqty{3 & -1 & 1 \\
            1 & 3 & 2\\
            -2 & -1 & 1}\\
            adj(A)=\bmqty{3 & 1 & -2 \\
            -1 & 3 & -1 \\
            1 & 2 & 1}\\
            \det{A}= 1(3) - (-1)(1) + 1(1)= 5\\
            A^{-1}=\frac{1}{5}\bmqty{3 & 1 & -2 \\
            -1 & 3 & -1 \\
            1 & 2 & 1}
        \end{gather*}
    \end{problem}
    \begin{problem}
        \textbf{1G 4}\\
        Referring to Problem above, solve the system\\
        \begin{gather*}
            \mqty{x_1 & - x_2 & + x_3 & = y_1 \\
            & x_2 & + x_3 & = y_2 \\
            -x_1 & -x_2 & + 2x_3 & = y_3}\\
            \text{Answer:}\\
            \mqty{x_1 & = \frac{3}{5}y_1 & +\frac{1}{5}y_2 & -\frac{2}{5}y_3 \\
            x_2 &= -\frac{1}{5}y_1 & +\frac{3}{5}y_2 & -\frac{1}{5}y_3 \\
            x_3 & = \frac{1}{5}y_1 & + \frac{2}{5}y_2 & + \frac{1}{5}y_3}
        \end{gather*}
    \end{problem}
    \begin{problem}
        \textbf{1G 5}\\
        Show that $(AB)^{-1} = B^{-1} A^{-1}$ by using the definition of inverse matrix.\\
        \begin{gather*}
            (B^{-1} A^{-1})(AB) = I\\
            (AB)(B^{-1} A^{-1}) = I\\
            \text{Therefore $(B^{-1} A^{-1})$ is inverse to $AB$ }
        \end{gather*}
    \end{problem}


    \section{Theorems about square systems, equations of planes}
    \begin{problem}
        \textbf{1H 3}
        \begin{itemize}
            \item a)For what value c following equation will have non trivial solution?\\
            \begin{gather*}
                \mqty{x_1 & - x_2 & + x_3 & = 0\\
                2x_1 & + x_2 & + x_3 & = 0 \\
                -x_1 & + cx_2 & + 2x_3 & =0}\\
                det=2-c +4 +1 +2c +1 = c +8\\
                \text{The system has a non trivial solution if the determinant is equal to zero}\\
                $c=-8$
            \end{gather*}
            \item b)For what c value will $\bmqty{2 & 1 \\0 & -1}\bmqty{x \\ y}=c \bmqty{x \\ y}$  have a non-trivial solution? (Write it as a system of homogenous equations)\\
            \begin{gather*}
                \mqty{(2-c)x & +cy & =0\\
                & (-1 - c)y & =0}\\
                det=(2 -c)(-1-c)=-2 -2c+c + c^2 = 0\\
                c=2 \text{ or } c=-1
            \end{gather*}
            \item c) For each vale of c i part a) find non-trivial solution to the corresponding system. (Interpret the equations as asking for a vector orthogonal to three given vectors, find it by using the cross product)\\
            \begin{gather*}
                \text{Vector orthogonal to first two equations}\\
                \bmqty{1 & -1 & 1} \cross \bmqty{2 & 1 & 1}=\bmqty{\vu{x_1} & \vu{x_2 } & \vu{x_3}\\
                1 & -1 & 1 \\
                2 & 1 & 1}=
                -2x_1 + x_2 +3 x_3 \\
                \text{This is orthogonal also to third vector because}\\
                \bmqty{-1 & -8 & 2} \vdot \bmaty{-2 & 1 & 3}=0
            \end{gather*}
        \end{itemize}
    \end{problem}
    \begin{problem}
        \textbf{1E 1}\\
        Find the equations of the following planes
        \begin{itemize}
            \item c) through $\bmqty{2 & -1 & -1}, \bmqty{2 & -1 & 2}, \bmqty{2 & -1 & 3}$\\
            \text{Just find vectors and calculate cross product}
        \end{itemize}
    \end{problem}
    \begin{problem}
        \textbf{1E 2}\\
        Find the dihedral angle between $2x - y + z =3$ and $x + y + 2z = 1$\\
        Find the angle between normal vectors using dot product between their unit vectors.\\
        \begin{gather*}
            \vu{a}=\frac{1}{\sqrty{6}}\bmqty{2 \\ -1 \\ 1} \\
            \vu{b} = \frac{1}{\sqrt{6}}\bmqty{1 \\ 1 \\ 2}\\
            \cos{\theta}=\frac{3}{6}=\frac{1}{2}\\
            \theta=\ang{60}
        \end{gather*}
    \end{problem}

\end{document}