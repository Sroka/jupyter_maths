%%
%% Author: maciejsrokowski
%% 27/07/2018
%%

% Preamble
\documentclass[11pt]{article}

% Packages
\usepackage{amsmath}
\usepackage{physics}
\usepackage{siunitx}
\usepackage{graphicx}
\usepackage{cancel}
\usepackage{a4wide}

% Document
\begin{document}

    \title{Problem Set 2}
    \maketitle

    \newtheorem{theorem}{Theorem}
    \newtheorem{definition}{Definition}
    \newtheorem{example}{Example}
    \newtheorem{problem}{Problem}

    \section{Matrices and inverse matrices}
    \begin{problem}
        \textbf{1F 5b}
        Find $A^2 ,A^3 , A^n $ if $A=\bmqty{1 & 1 \\ 0 & 1}$
        \begin{gather*}
            A^2 = \bmqty{1 & 2 \\ 0 & 1}\\
            A^3 = \bmqty{1 & 3 \\ 0 & 1}\\
            \text{Guess:}\\
            A^n = \bmqty{1 & n \\ 0 & 1}\\
        \end{gather*}
    \end{problem}
    \begin{problem}
        \textbf{1F 8a}
        If \[
               A \bmqty{1 \\ 0 \\ 0} = \bmqty{2 \\ 3 \\ 1 },
               A \bmqty{0 \\1 \\ 0 } = \bmqty{-1 \\ 0 \\ 4 },
               A \bmqty{0 \\ 0 \\ 1} = \bmqty{1 \\ 1 \\ -1}
        \], what is the matrix $3 \cross 3 $ A\\
        \begin{gather*}
            A = \bmqty{
            2 & -1 & 1 \\
            3 & 0 & 1 \\
            1 & 4 & -1
            }
        \end{gather*}
    \end{problem}
    \begin{problem}
        \textbf{1G 3}\\
        \begin{gather*}
            A = \bmqty{1 & -1 & 1\\
            0 & 1 & 1 \\
            -1 & -1 & 2}\\
            b = \bmqty{ 2 \\ 0 \\ 3 }\\
            \text{Solve $Ax=b$ by finding $A^{-1}$}\\
            \det{A}=\
        \end{gather*}
    \end{problem}
    \begin{problem}
        \textbf{1G 4}
    \end{problem}
    \begin{problem}
        \textbf{1G 5}
    \end{problem}


    \section{Theorems about square systems, equations of planes}
    \begin{problem}
        \textbf{1H 3}
        \begin{itemize}
            \item a)
            \item b)
            \item c)
        \end{itemize}
    \end{problem}
    \begin{problem}
        \textbf{1E 1}
        \begin{itemize}
            \item c)
            \item d)
        \end{itemize}
    \end{problem}
    \begin{problem}
        \textbf{1E 2}
        \begin{problem}
            \textbf{1E 6}
        \end{problem}
    \end{problem}
\end{document}