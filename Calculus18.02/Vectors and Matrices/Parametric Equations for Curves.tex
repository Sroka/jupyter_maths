%%
%% Author: maciejsrokowski
%% 13/08/2018
%%

% Preamble
\documentclass[11pt]{article}

% Packages
\usepackage{a4wide}
\usepackage{amsmath}
\usepackage{physics}
\usepackage{siunitx}
\usepackage{graphicx}
\usepackage{cancel}

% Document
\begin{document}

    \title{Parametric equations for curves}
    \maketitle

    \newtheorem{theorem}{Theorem}
    \newtheorem{definition}{Definition}
    \newtheorem{example}{Example}
    \newtheorem{problem}{Problem}


    \section{Equations of lines}

    We can think of a line as an intersection fo two planes.\\
    Another way to think of a line is to represent it as a moving point or parametric equation.

    \begin{example}
        Line through points $Q_0 = \bmqty{-1 & 2 & 2}$ and $Q_1 = \bmqty{1 & 3 & -1}$ can be represented with the use of parameter $t$. So a line is $Q(t)$, a moving point where $Q_0$ is when $t=0$.\\
        Q: What is the position at time t?\\
        A: $\va{Q_0 Q(t)} = t \va{(Q_0 Q_{1)}}$, assuming that the movement is linear\\

        \begin{gather}
            \va{Q_0 Q(t)} = t \va{(Q_0 Q_{1)}}= t \bmqty{2 & 1 & -3}\\
            \text{So the curve will be described by equations:}\\
            \mqty{x(t) & +1 & = t2 \\
            y(t) & - 2 & =t \\
            z(t) & -2 & = -3t}\\
            Q(t) = Q_0 + t\va{Q_0 Q_1}
        \end{gather}
    \end{example}

    \begin{example}
        \left. \begin{array} { l } { \text { Give parametric equations for } x , y , z , \text { on the line through } ( 1,1,2 ) \text { in a direction parallel } }
                   \\ { \text { to } \langle 2 , - 3 , - 1 \rangle }
        \end{array} \right.\\

        Remember that for each coordinate the parametric equation consists of the constant by which it is moved from the origin and parameter
        \begin{gather*}
            x-1 = 2t \\
            y-1 = -3t\\
            y - 2 = -t
        \end{gather*}
    \end{example}

    \begin{example}
        \left. \begin{array} { l } { \text { Give parametric equations for the intersection of the planes } x + y + z = 1 \text { and } }
                   \\ { x + 2 y + 3 z = 2 }
        \end{array} \right
        \\

        The line of intersection is perpendicular to both normals (to the planes)
        \begin{gather*}
            v = \bmqty{1 & 1 & 1 } \cross \bmqty{ 1 & 2 & 3 } = \bmqty{ i & j & k \\
            1 & 1 & 1 \\
            1 & 2 & 3 }
            =\bmqty{1 & -2 & 1}
        \end{gather*}
        We still need to take into account the fact that those plane equations are not a homogenous linear system. For this just make one of the variables 0 and calculate a single point.\\
        \begin{gather*}
            x = 0\\
            \mqty{ y & + z & = 1\\
            2y & + 3z & = 2}\\
            y = 1\\
            z = 0\\
            \text{This line exists for point where above holds}\\
            \text{Final answer:}\\
            x = t\\
            y = -2t +1 \\
            z = t
        \end{gather*}
    \end{example}

    \section{Intersection of a line and a plane}

    Consider the plane $x + 2y + 4z = 7$ . Where does the line though $Q_0=\bmqty{-1 & 2 & 2 }$ and $Q_1 = \bmqty{ 1 & 3 & -1}$ intersects the plane?\\

    In example 1 we calculated the line between $Q_0$ and $Q_1$ as $(-1 + 2t) + 2(2 + t) + 4 ( 2- 3t)$
    \begin{gather*}
        x(t) + 2y(t) + 4z(t) =  (-1 + 2t) + 2(2 + t) + 4 ( 2- 3t) = -8t + 11 = 7\\
        t = \frac{1}{2}\\
        Q(\frac{1}{2}) = \bmqty{ 0 & \frac{5}{2} * \frac{1}{2}}
    \end{gather*}

    If we don't get a solution it means that the line is parallel to the plane

    \begin{example}
        \left. \begin{array} { l } { \text { Find the intersection of the line through the points } ( 1,3,0 ) \text { and } ( 1,2,4 ) \text { with the plane } }
                   \\ { \text { through the points } ( 0,0,0 ) , ( 1,1,0 ) \text { and } ( 0,1,1 ) }
        \end{array} \right.\\

        \begin{gather*}
            \text{Find the line direction} \\
            \bmqty{(1 - 1) & (2 - 3) & (4 - 0)} = \bmqty{0 & -1 & 4}\\
            \text{Parametrize it so that it passes thorugh the point}\\
            x = 1 \\
            y = -t + 3 \\
            z = 4t \\
            \text{Find the plane by cross product}\\
            \bmqty{ i & j & k\\
            1 & 1 & 0 \\
            0 & 1 & 1 }=
            \bmqty{ 1 & -1 & 1}\\
            \text{Check if they intersect}\\
            1(1) -1(-t + 3) + 1(4t) = 0\\
            1 +t -3 + 4t = 0\\
            t = \frac{2}{5}\\
            \pmqty{x & y & z} = \bmqty{1 & 3\frac{3}{5} & 1\frac{3}{8}}
        \end{gather*}
    \end{example}

    \section{Cycloid}
    Cycloid is a curve created by the point on a rolling circle.\\

    What happened near the bottom point of a cycloid? Take length unit = radious\\
    \begin{gather*}
        \mqty{x(\theta) = \theta - \sin{\theta} \\
        y(\theta) = 1 - \cos{\theta}}
    \end{gather*}
    When w approximate $\sin{\theta} = \theta$ and $\cos{\theta} = 1$ we get $0 = 0$. That means that the approximation is not good enough. Maybe we can use Taylor expansion.

    \begin{gather*}
        \sin{\theta} \approx \theta - \frac{\theta^3 }{6}\\
        \cos{\theta} \approx 1 - \frac{\theta^2}{2}
    \end{gather*}
    So that:
    \begin{gather*}
        x(\theta) = \theta - (\theta - \frac{\theta^3}{6}) \\
        y(\theta) = 1 - (1 - \frac{\theta^2}{2})
    \end{gather*}
    When $\theta$ becomes very small $x << y$\\
    \begin{gather*}
        \frac{y}{x} \approx \frac{\frac{\theta^2}{2}}{\frac{\theta^3}{6}}
        = \frac{3}{\theta}
    \end{gather*}
    For $\theta \rightarrow 0$ value of a derivative at this point becomes $\infty$

    \section{Velocity and accelaration}
    An easy way to think about parametric equation is to imagine the position of the point as an end of a vector $OP$ where O is and origin and P a position of a point.\\
    We can also analyze the parametric equation in terms of its derivatives so velocity and acceleration:
    \begin{gather*}
        \va{V} = \frac{dr}{dt} = \bmqty{\frac{dx}{dt} & \frac{dy}{dt} & \frac{dz}{dt}}
    \end{gather*}

    \section{Product rule for vector derivatives}
    If r1(t) and r2(t) are two parametric curves show the product rule for derivatives holds for the dot product.\\
    Answer: This will follow from the usual product rule in single variable calculus. Lets assume the curves are in the plane. The proof would be exactly the same for curves in space. We want to prove that:
    \begin{gather*}
        \frac { d \left( \mathbf { r } _ { 1 } \cdot \mathbf { r } _ { 2 } \right) } { d t } = \mathbf { r } _ { 1 } ^ { \prime } \cdot \mathbf { r } _ { 2 } + r _ { 1 } \cdot \mathbf { r } _ { 2 } ^ { \prime }
    \end{gather*}

    \text { Let } \mathbf { r } _ { 1 } = \left\langle x _ { 1 } , y _ { 1 } \right\rangle \quad \text { and } \quad \mathbf { r } _ { 2 } = \left\langle x _ { 2 } , y _ { 2 } \right\rangle . \text { We have }

    \begin{gather*}
        \mathbf { r } _ { 1 } \cdot \mathbf { r } _ { 2 } = x _ { 1 } x _ { 2 } + y _ { 1 } y _ { 2 }
    \end{gather*}

    Taking derivatives using the product rule from single variable calculus, we get:

    \begin{gather*}
        \left.\begin{aligned}
                  \frac { d \left( \mathbf { r } _ { 1 } \cdot \mathbf { r } _ { 2 } \right) } { d t } & = \frac { d \left( x _ { 1 } x _ { 2 } + y _ { 1 } y _ { 2 } \right) } { d t } \\ & = x _ { 1 } ^ { \prime } x _ { 2 } + x _ { 1 } x _ { 2 } ^ { \prime } + y _ { 1 } ^ { \prime } y _ { 2 } + y _ { 1 } y _ { 2 } ^ { \prime } \\ & = \left( x _ { 1 } ^ { \prime } x _ { 2 } + y _ { 1 } ^ { \prime } y _ { 2 } \right) + \left( x _ { 1 } x _ { 2 } ^ { \prime } + y _ { 1 } y _ { 2 } ^ { \prime } \right) \\ & = \left\langle x _ { 1 } ^ { \prime } , y _ { 1 } ^ { \prime } \right\rangle \cdot \left\langle x _ { 2 } , y _ { 2 } \right\rangle + \left\langle x _ { 1 } , y _ { 1 } \right\rangle \cdot \left\langle x _ { 2 } ^ { \prime } , y _ { 2 } ^ { \prime } \right\rangle \\ & = \mathbf { r } _ { 1 } ^ { \prime } \cdot \mathbf { r } _ { 2 } + \mathbf { r } _ { 1 } \cdot \mathbf { r } _ { 2 } ^ { \prime }
        \end{aligned} \right.
    \end{gather*}


    \begin{example}
        \left. \begin{array} { l } { 1 . \text { If } \mathbf { r } _ { 1 } ( t ) \text { and } \mathbf { r } _ { 2 } ( t ) \text { are two parametric curves show the product rule for derivatives holds } }
                   \\ { \text { for the cross product. } }
        \end{array} \right.\\

        \begin{gather*}
            r_1 (t) \cross r_2 (t) =
            \vu{x} (r_{1y} r_{2z} - r_{1z} r_{2y})
            - \vu{y} (r_{1x} r_{2z} - r_{1z} r_{2x})
            + \vu{z} (r_{1y} r_{2z} - r_{1z} r_{2y})\\
            \frac { d \left( \mathbf { r } _ { 1 } \times \mathbf { r } _ { 2 } \right) } { d t } = \left\langle y _ { 1 } ^ { \prime } z _ { 2 } + y _ { 1 } z _ { 2 } ^ { \prime } - z _ { 1 } y _ { 2 } ^ { \prime } , z _ { 1 } ^ { \prime } x _ { 2 } + x _ { 1 } ^ { \prime } z _ { 2 } - x _ { 1 } z _ { 2 } ^ { \prime } , x _ { 1 } ^ { \prime } y _ { 2 } + x _ { 1 } y _ { 2 } ^ { \prime } - y _ { 1 } ^ { \prime } x _ { 2 } - y _ { 1 } x _ { 2 } ^ { \prime } \right\rangle\\
            = \left\langle \left( y _ { 1 } ^ { \prime } z _ { 2 } - z _ { 1 } ^ { \prime } y _ { 2 } \right) + \left( y _ { 1 } z _ { 2 } ^ { \prime } - z _ { 1 } y _ { 2 } ^ { \prime } \right) , \left( z _ { 1 } ^ { \prime } x _ { 2 } - x _ { 1 } ^ { \prime } z _ { 2 } \right) + \left( z _ { 1 } x _ { 2 } ^ { \prime } - x _ { 1 } z _ { 2 } ^ { \prime } \right) , \left( x _ { 1 } ^ { \prime } y _ { 2 } - y _ { 1 } ^ { \prime } x _ { 2 } \right) + \left( x _ { 1 } y _ { 2 } ^ { \prime } - y _ { 1 } x _ { 2 } ^ { \prime } \right) \right\rangle\\
            = \left\langle x _ { 1 } ^ { \prime } , y _ { 1 } ^ { \prime } , z _ { 1 } ^ { \prime } \right\rangle \times \left\langle x _ { 2 } , y _ { 2 } , z _ { 2 } \right\rangle + \left\langle x _ { 1 } , y _ { 1 } , z _ { 1 } \right\rangle \times \left\langle x _ { 2 } ^ { \prime } , y _ { 2 } ^ { \prime } , z _ { 2 } ^ { \prime } \right\rangle\\
            = \mathbf { r } _ { 1 } ^ { \prime } \times \mathbf { r } _ { 2 } + \mathbf { r } _ { 1 } \times \mathbf { r } _ { 2 } ^ { \prime }
        \end{gather*}

    \end{example}
\end{document}