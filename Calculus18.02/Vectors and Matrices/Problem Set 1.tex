%%
%% Author: maciejsrokowski
%% 15/07/2018
%%

% Preamble
\documentclass[11pt]{article}

% Packages
\usepackage{a4wide}
\usepackage{amsmath}
\usepackage{physics}
\usepackage{siunitx}
\usepackage{graphicx}
\usepackage{cancel}

\newtheorem{excercise}{Excercise}

% Document
\begin{document}

    \title{Problem Set 1}
    \maketitle

    \section{Part I}

    \begin{excercise}
        \textbf{1A,6}\\
        A small plane wishes to fly due north at 200 mph (as seen from the ground), in a wind blowing from the northeast at 50 mph. Tell with what vector velocity in the air it should travel (give the i j -components).\\
        \\
        W - wind vector\\
        P - plane vector\\
        \begin{align}
            \va{P}=\bmqty{x & y}\\
            \va{W}=\bmqty{-50\cos{\ang{45}} & -50\sin{\ang{45}}}\\
            \va{W}=\bmqty{-25\sqrt{2} & -25\sqrt{2}}
            \text{The equality must hold:}\\
            \va{W} + \va{P} = \bmqty{200 & 0}\\
            -25\sqrt{2} + x = 200\\
            x = 200 + 25\sqrt{2}\\
            -25\sqrt{2} + y = 0\\
            y = 25\sqrt{2}\\
            \va{P} = \bmqty{200 + 25\sqrt{2} & 25\sqrt{2}}
        \end{align}

    \end{excercise}

    \begin{excercise}
        \textbf{1A,7}\\
        Let $\va{A}=ai+bj$ be a plane vector; find in terms of a and b the vectors $\va{A\prime}$ and $\va{A\prime\prime}$ resulting from rotating $\va{A}$ by \ang{90}

        \begin{itemize}
            \item a) clockwise\\
            $\va{A\prime}=bi - aj$\\
            $\va{A\prime\prime}=-ai - bj$\\
            \item b) counterclockwise\\
            $\va{A\prime}=-bi + aj$\\
            $\va{A\prime\prime}=-ai - bj$\\
            (Hint: make A the diagonal of a rectangle with sides on the x and y-axes, and rotate the whole rectangle.)
            \item c) Let $i\prime = (3i +4j)/5$. Show that $i\prime$ is a unit vector, and use the first part of the exercise to find a vector $j\prime$ such that $i\prime$, $j'$ forms a right-handed coordinate system\\
            $\abs{i\prime}=\sqrt{\frac{3}{5}^2 + \frac{4}{5}^2}=1$\\
            $i'=-\frac{4}{5}j + \frac{3}{5}j$
        \end{itemize}
    \end{excercise}

    \begin{excercise}
        \textbf{1A,9}\\
        Prove using vector methods (without components) that the line segment joining the midpoints of two sides of a triangle is parallel to the third side and half its length. (Call the two sides A and B.)\\
        \begin{align}
            \va{A}, \va{B}, \va{C}\text{ - edges of a triangle}\\
            \va{D}\text{ - line joining midpoints of A and B}\\
            \text{ In a triangle:}\\
            \va{A} + \va{B} = \va{C}\\
            \frac{1}{2}\va{A} + \frac{1}{2}\va{B} = \va{D}\\
            \va{A} + \va{B} = 2\va{D}\\
            \va{C} = 2\va{D}
        \end{align}
    \end{excercise}

    \begin{excercise}
        \textbf{1B, 2}
        Tell for what values of c the vectors ci+2j-k and i-j+2k will
        \begin{itemize}
            \item a) be orthogonal
            \begin{align}
                \bmqty{ci & 2j & -k}\vdot\bmqty{i & -j & 2k}=0\\
                c-2-2=0\\
                c=4
            \end{align}
            \item b) form an acute angle
            \begin{align}
                c>0
            \end{align}
        \end{itemize}
    \end{excercise}

    \begin{excercise}
        \textbf{1B, 5b}
        Find the component of the force $\va{F}=2i-2j+k$ in the direction of the vector $3i+2j-6k$
        \begin{align}
            \text{Find unit vector} & \\
            \abs{3i+2j-6k}=\sqrt{3^2 + 2^2 + (-6)^2}=7\\
            \vu{u}=\mbqty{\frac{3}{7}i + \frac{2}{7}j - \frac{6}{7}k}\\
            \text{Use dot product}\\
            \va{F}\vdot\vu{u}=\frac{6}{7}i - \frac{4}{7}j + \frac{6}{7}=
        \end{align}
    \end{excercise}

    \begin{excercise}
        \textbf{1B, 12}\\
        Prove using vector methods (without components) that an angle inscribed in a semicircle is a right angle.\\
        \begin{gather}
            \text{A, B - semi-circle ends}\\
            \text{M - point on a semi-circle}\\
            \text{O - semi-circle center}\\
            \va{AO} + \va{OM} = \va{AM}\\
            \va{BO} + \va{OM} = \va{BM}\\
            \text{Inner product is distributive:}\\
            \va{AM}\vdot\va{BM}=(\va{AO} + \va{OM})\vdot(\va{BO} + \va{OM})\\
            \va{AM}\vdot\va{BM}
            =\va{AO}\vdot\va{BO}
            + \va{AO}\vdot\va{OM}
            + \va{OM}\vdot\va{BO}
            + \va{OM}\vdot\va{OM}\\
            \text{Since $\va{AO} = -\va{BO}:$}\\
            \va{AM}\vdot\va{BM}
            =-\va{BO}\vdot\va{BO}
            - \cancel{\va{BO}\vdot\va{OM}}
            + \cancel{\va{OM}\vdot\va{BO}}
            + \va{OM}\vdot\va{OM}\\
            \text{Since:}\\
            \va{BO}\vdot\va{BO}=\abs{\va{BO}}^2 \\
            \va{OM}\vdot\va{OM}=\abs{\va{OM}}^2 \\
            \text{and}\\
            \abs{\va{OM}}=\abs{\va{BO}}\\
            \text{then}\\
            \va{AM}\vdot\va{BM}=0
        \end{gather}

    \end{excercise}

    \begin{excercise}
        \textbf{1B, 13}\\
        Prove the trigonometric formula $\cos{(\theta_1 - \theta_2 )}=\cos{\theta_1}\cos{\theta_2} + \sin{\theta_1}\sin{\theta_2}$\\
        (Hint: consider two unit vectors making angles $\theta_1$ and $\theta_2$ with the positive x-axis)\\
        \begin{align}
            \vu{u_1} & = \cos{\theta_1}i + \sin{\theta_2}j\\
            \vu{u_2} & = \cos{\theta_2}i + \sin{\theta_2}j\\
            \cos{(\theta_1 - \theta_2)} & = \frac{\vu{u_1}\vdot\vu{u_2}}{\abs{\vu{u_1}}\abs{\vu{u_2}}}\\
            & = \cos{\theta_1}\cos{\theta_2} + \sin{\theta_1}\sin{\theta_2}
        \end{align}
    \end{excercise}

    \begin{excercise}
        \textbf{1C, 2}\\
        Calculate $\vmqty{-1 & 0 & 4 \\ 1 & 2 & 2 \\ 3 & -2 & -1}$ using the Laplace expansion by cofactors of:
        \begin{itemize}
            \item a) the first row\\
            \begin{align}
                \vmqty{-1 & 0 & 4 \\ 1 & 2 & 2 \\ 3 & -2 & -1}
                & =-1\vmqty{2 & 2 \\ -2 & -1}
                -0\vmqty{1 & 2 \\ 3 & -1 }
                +4\vmqty{1 & 2 \\ 3 & -2}\\
                & = (-1)(-2+4) - 0(-1 -6) + 4(-2-6)\\
                & = -34
            \end{align}
            \item b) the first column\\
            \begin{align}
                \vmqty{-1 & 0 & 4 \\ 1 & 2 & 2 \\ 3 & -2 & -1}
                & = -1 \vmqty{2 & 2 \\ -2 & -1}
                -1\vmqty{0 & 4 \\ -2 & -1}
                +3\vmqty{0 & 4 \\ 2 & 2}\\
                & = -1(-2 + 4)
                -1(0+8)
                +3(0-8)\\
                & = -2 -8 - 24\\
                & = -34
            \end{align}
        \end{itemize}
    \end{excercise}

    \begin{excercise}
        \textbf{1C, 5a}\\
        Show that the value of a $2\cross2$ determinant is unchanged if you add to the second row a scalar multiple of the first row\\
        \begin{gather}
            \vmqty{a_1 & a_2 \\ b_1 & b_2} = a_1 b_2 - a_2 b_1 \\
            \vmqty{a_1 & a_2 \\ b_1 + ca_1 & b_2 + ca_2}=
            a_1 (b_2 + ca_2 ) - a_2 (b_1 + ca_1 )
            = a_1 b_2 + \cancel{a_1 c a_2} - a_2 b_1 - \cancel{a_2 c a_1}
        \end{gather}
    \end{excercise}

    \begin{excercise}
        \textbf{1D, 2}\\
        Find the area of the triangle in space having its vertices at the points:
        \begin{gather}
            P:(2, 0, 1)\\
            Q: (3, 1, 0)\\
            R: (-1, 1, -1)\\
            \va{PQ}=1i 1j -1k\\
            \va{PR}=-3i 1j -2k\\
            \va{PQ}\cross\va{PR}
            =\vmqty{
            i & j & k \\
            1 & 1 & -1 \\
            -3 & 1 & -2
            }
            =i(-1) - j(-5) + k(4)= -1i +5j + 4k\\
            area=\frac{1}{2}\abs{\va{PQ} \cross \va{PR}}=\sqrt{1 + 25 + 16}=\frac{1}{2}\sqrt{42}
        \end{gather}
    \end{excercise}

    \begin{excercise}
        \textbf{1D, 5}\\
        What can you conclude about A and B
        \begin{itemize}
            \item if $\abs{A \cross B}=\abs{A}\abs{B}$\\
            A and B are orthogonal
            \item if $\abs{A \cross B}=A\vdot B$\\
            $\abs{A}\abs{B}\sin{\theta}=\abs{A}\abs{B}\cos{\theta}$\\
            $\sin{\theta}=\cos{\theta}$\\
            \theta=\frac{\pi}{4}

        \end{itemize}
    \end{excercise}

    \begin{excercise}
        \textbf{1D, 7}\\
        Find the volume of the tetrahedron having vertices at the four points:\\
        P:(1,0,1)\\
        Q:(-1,1,2)\\
        R:(0,0,2)\\
        S:(3,1,-1)\\
        (Hint: volume of tetrahedron = $\frac{1}{6}$(volume of parallelepiped with 3 same conterminous edges))\\
        \begin{gather}
            \va{PQ}=\bmqty{-2 & 1 & 1}\\
            \va{PR}=\bmqty{-1 & 0 & 1}\\
            \va{PS}=\bmqty{2 & 1 & -2}\\
            \det{\va{PQ}, \va{PR}, \va{PS}}=\va{PQ}\vdot(\va{PR}\cross\va{PS})\\
            \va{PR}\cross\va{PS}=\bmqty{-3 & -4 & -1}\\
            \va{PQ}\vdot(\va{PR}\cross\va{PS})
            =6 -4 -1 = 1\\
            V=\frac{1}{6}\det{\va{PQ},\va{PR}\va{PS}}=\frac{1}{6}
        \end{gather}
    \end{excercise}

    \section{Part II}

    \begin{excercise}
        Find the dihedral angle between two faces of a regular tetrahedron.\\
        \begin{gather}
            \text{Describe tetrahedron by 4 points:}\\
            P:(0, 0 ,0)\\
            Q:(1, 1, 0)\\
            R:(0, 1, 1)\\
            S:(1, 0, 1)\\
            \text{Set one point M between points Q and R}\\
            \va{AM}\vdot\va{PM}=\abs{\va{AM}}\abs{\va{PM}}\cos{\theta}\\
            \abs{\va{PM}}=\abs{\va{PQ}}\cos{\ang{30}}=\frac{\sqrt{3}}{2}\abs{\va{PQ}}\\
            \abs{\va{AM}}=\abs{\va{PM}}\\
            \cos{\theta}=\frac{\va{AM}\vdot\va{PM}}{\abs{\va{AM}}\abs{\va{PM}}}\\
            \theta\approx\ang{70.5}
        \end{gather}
    \end{excercise}

    \begin{excercise}

        \begin{itemize}
            \item a) Show that the 'polarization identity' $\frac{1}{4}(\abs{u+v}^2 - \abs{u-v}^2)=u\vdot v$ holds for any two n-vectors u and v. (Use vector algebra, not components.)\\
            \begin{gather}
                (u + v)^2 = u^2 + 2u\vdot v + v^2 \\
                (u - v)^2 = u^2 - 2u\vdot v + v^2 \\
                (u + v)^2 - (u - v)^2 = 4u \vdot v
                \frac{1}{4}((u + v)^2 - (u - v)^2 ) = u \vdot v
            \end{gather}
            \item b) Given two non-zero vectors u and v ,give the formula for the unit vector which bisects the (smaller) angle between u and v.\\
            (Use the notation $\vu{u}$ for the unit vector in the u - direction)\\
            \begin{gather}
                      \vu{r}=\frac{u}{\abs{v}} + \frac{v}{\abs{v}}
            \end{gather}
        \end{itemize}
    \end{excercise}
\end{document}