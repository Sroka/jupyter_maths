%%
%% Author: maciejsrokowski
%% 26/08/2018
%%

% Preamble
\documentclass[11pt]{article}

% Packages
\usepackage{amsmath}
\usepackage{physics}
\usepackage{siunitx}
\usepackage{graphicx}
\usepackage{cancel}
\usepackage{a4wide}

% Document
\begin{document}

    \title{Problem Set 3}
    \maketitle

    \newtheorem{theorem}{Theorem}
    \newtheorem{definition}{Definition}
    \newtheorem{example}{Example}
    \newtheorem{problem}{Problem}

    \section{Parametric equations for lines and curves}

    \begin{problem}
        \textbf{1E 3c}\\
        Find in parametric form the equations for all lines passing through $(1, 1, 1)$ and lying in the plane $x + 2y - z = 2$\\
        The lines parallel to the normal vector of the plane have a vector form:
        \[\bmqty{ax & by & cz}\]
        where:
        \[a + 2b - c =0\]
        They are also passing through a point $(1, 1, 1)$ when:
        \[\bmqty{ax+1 & by +1 & -cz +1}\]
    \end{problem}

    \begin{problem}
        \textbf{1E 4}\\
        Where does the line through $( 0,1,2 )$ and $( 2,0,3 )$ intersect the plane $x + 4 y + z = 4$ ?\\
        The equation of the line can be described by parametric equations:\\
        \begin{gather*}
            x = 2t\\
            y = -t + 1\\
            z = t + 2
        \end{gather*}
        It intersects the plane in the point:\\
        \begin{gather*}
            2t + 4(-t + 1) + t + 2 = 4\\
            t = 2
        \end{gather*}
        So the point is:
        \begin{gather*}
            \bmeqty{ 4 & - 1 & 4}
        \end{gather*}
    \end{problem}

    \begin{problem}
        \textbf{1I 3}\\
        Describe the motions given by each of the following position vector functions, as t goes from $-\infty$ to $\infty$. In each case, give the xy-equation of the curve along which P travels, and tell what part of the curve is actually traced out by P .\\
        \begin{enumerate}
            \item a) $\mathbf { r } = 2 \cos ^ { 2 } t \mathbf { i } + \sin ^ { 2 } t \mathbf { j }$\\
            \begin{gather*}
                x = 2\cos^2{t}\\
                y = \sin^2{t}\\
                y = 1 - \cos^2 {t} \\
                y = 1 - \frac{1}{2} x
            \end{gather*}
            \item b) $\mathbf { r } = \cos 2 t \mathbf { i } + \cos t \mathbf { j }$\\
            \begin{gather*}
                x = \cos{2t} \\
                y = \cos{t} \\
                x = 2\cos^2 {t} - 1\\
                x = 2y^2 - 1
            \end{gather*}
        \end{enumerate}
    \end{problem}

    \begin{problem}
        \textbf{1I 5}\\
        A string is wound clockwise around the circle of radius a centered at the origin O; the initial position of the end P of the string is (a, 0). Unwind the string, always pulling it taut (so it stays tangent to the circle). Write parametric equations for the motion of P .
        (Use vectors; express the position vector OP as a vector function of one variable.)\\
        Use the rotation angle $\theta$ as a variable\\
        Let the Q be the point on the circle that is tangent to the string
        \begin{gather*}
            \va{OP} = \va{OQ} + \va{QP} \\
            \va{OQ} = \bmqty{a\cos{\theta} & a\sin{\theta}} \\
            \va{QP} = \bmqty{a\theta \cos{90 - \theta} & -a\theta \sin{90 - \theta}}\\
            \text{Use trig identities of reflected angle}\\
            \sin{\pi - \theta} = \cos{\theta}\\
            \cos{\pi - \theta} = \sin{\theta}\\
            \va{OP} = \bmqty{a(\cos{\theta} + \theta\sin{\theta}) & a(\sin{\theta} - \theta\cos{\theta})}
        \end{gather*}

    \end{problem}

    \begin{problem}
        \textbf{1J 2}\\
        Let $\va{OP} = \frac { 1 } { 1 + t ^ { 2 } } \mathbf { i } + \frac { t } { 1 + t ^ { 2 } } \mathbf { j }$ be the position vector for the motion.
        \begin{enumerate}
            \item a) Calculate $v, \vmqty{\frac{ds}{dt}}, T $
            \begin{align*}
                v & = \dv{}{t}\sqrt{(\frac{1i}{1 + t^2})^2 + (\frac{tj}{1 + t^2})^2}\\
                & = \dv{}{t}\frac{\sqrt{i^2 + t^2j^2}}{1 + t^2} \\
                & = \dv{}{t}\frac{i - tj}{1 + t^2}\\
                & = \frac{- 2it}{(1 + t^2)^2} + \frac{j}{1 + t^2} - \frac{2t^2 j}{(1 + t^2)^2}\\
                & = \frac{- 2ti + (1 - t^2)j}{(1 + t^2)^2}
            \end{align*}
            \begin{align*}
                \vmqty{v} & = \sqrt{(\dv{}{t}\frac{1}{1 + t^2})^2 + (\dv{}{t}\frac{t}{1 + t^2})^2} \\
                & = \sqrt{(\frac{-2t}{(1 + t^2)^2})^2 + (\frac{1 + t^2 - 2t^2}{(1 + t^2)^2})^2}\\
                & = \frac{\sqrt{4t^2 + 1 + t^4}}{(1 + t^2)^2}\\
                & = \frac{\sqrt{(t^2 +1)^2}}{(1 + t^2)^2}\\
                & = \frac{1}{1 + t^2}\\
            \end{align*}

            \begin{gather*}
                T = \frac{-2ti+(1-t^2)j}{1+t^2}
            \end{gather*}
            \item b) At what point is the speed greatest? Smallest?\\
            When t equals 0\\
            \item c) Find the xy-equation of the curve along which the point P i moving and describe it geometrically
            \begin{gather*}
                y = tx\\
                x = \frac{1}{1 + t^2}\\
                t = \sqrt{\frac{1}{x} -1}\\
                y = x\sqrt{\frac{1}{x} -1}\\
                y^2 = x - x^2 \\
                y^2 + x^2 -x =0
            \end{gather*}
        \end{enumerate}
    \end{problem}
\end{document}